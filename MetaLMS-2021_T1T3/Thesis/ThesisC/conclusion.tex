\chapter{Conclusion}\label{ch:conclusion}

\section{Accounts and Enrollments}

\subsection{What would you have done differently}
json web vs oauth
account types

\subsection{Future work}
student un-enroll
account types
upgrade to oauth
staff/admin manage student accounts
export/import course enrollment data (csv of students)
download account information - privacy feature
delete account - privacy feature
\section{Topic Tree}
The aims of this thesis as discussed in the introduction, to allow course instructors to easily curate and manage content. The topic tree allows instructors to do just this, by providing more structure to content, and by providing a method to create and manage topic groups and topics themselves. Instructors can easily view prerequisites between topics and reuse content by cloning topics from other topic groups and setting specific topics as prerequisites for other topics. All of these features improve the reusability and management of content for instructors.\\

The main contributions of the topic tree include:
\begin{enumerate}
    \item A new method of creating and managing courses with the concept of topics and topic groups;
	\item Providing further structure to content by organising them by topics;
	\item Improving reusability by allowing topics in other topic groups to be set as prerequisites and by introducing the cloning feature;
    \item Providing adaptability to many learning models with files separated into four key sections;
    \item A functioning proof of concept topic tree that is well integrated into an overall learning management system.
\end{enumerate}

\subsection{Challenges}
The main challenge of the topic tree included developing the graph interface. This interface uses the D3 library, which allows for greater flexibility in developing data graphs, but with little support or documentation online. This made it quite difficult to develop the topic tree.\\

Other challenges included working with the rest of the team to integrate the topic tree with the rest of the system, as the rest of the team needed to know how topics and topic groups worked early on to develop their own features. Setting up topics and topic groups in the database was also quite difficult, due to the graph nature of the topic tree, as all this data had to be collated and loaded into the topic tree on first load with minimal impact to performance. Finally, some features listed in future work would involve working heavily with the rest of the team, such as tracking user progress required enrolment to be completely finished to retrieve students' enrolment data, and disciplines involved changing the database structure significantly which many API endpoints depended on, and so were not completed by the end of the thesis. \\

\subsection{Future Work}
Future work for the topic tree includes:

\begin{enumerate}
    \item Third party integration with YouTube, The Box, Moodle, etc.;
    \item Methods to export topic and topic group data to other platforms;
    \item Disciplines i.e. grouping topic groups together to improve performance and reduce information overload, especially as different schools or faculties mostly do not have prerequisites from other faculties;
    \item Graph performance and smoothness improvements;
    \item Student progress shown on the topic tree.
\end{enumerate} 


\section{Course Pages}

\subsection{What would you have done differently}
The main things that would have been done differently are:
\begin{itemize}
    \item Focusing more on the implementation of the course selection page
    \item Possibly restructuring the way the pages were implemented
    \item Restructuring some of the specifications for the backend API endpoints
\end{itemize}
The course selection page was added in Thesis B, it was after realising that a user would need a page to navigate to different courses.
This crucial element of the MetaLMS should have been emphasised in the development as it would be a central area of navigation.
Other things that could be done would be to improve the performance or allow for a less coupled system and provide more flexibility.

\subsection{Future Work}
Future features that could be implemented are:
\begin{itemize}
    \item Making the widgets bar more customisable
    \item Implementing the course outline page
    \item Adding more features to the course selection page
\end{itemize}
The current design of the widgets bar has 2 features, the calendar and reminders section.
In future implementation, more features could be added and the widgets bar could be made to be more customisable and allow for users to choose what widgets to have on the widgets bar.
The course selection page could also include more features and customisability as well.
The features in the course selection page are separated into boxes which could possibly allow for a more flexible design.
\section{Lectures and Tutorials}
The contributions of the lectures and tutorial component include:
\begin{enumerate}
    \item Allowing users to access lectures and tutorial files
    \item Users to view lectures or tutorials videos 
    \item Assisting users in searching for files related to lectures and tutorials
\end{enumerate}

\subsection{What would you have done differently}
There would be some research on how to incorporate YouTube's API to render the lecture videos and playlists within the lectures and tutorials component itself. Additionally, an early discussion should have been made to determine which feature should store relevant files to the course in order to avoid overlapping features. 

\subsection{Future Work}
Some future work for the lectures and tutorials components include:
\begin{enumerate}
    \item Visualise playlist of lecture/tutorial videos on the Meta LMS
    \item Allow lecturers to start/end lecture from the lectures/tutorials page
\end{enumerate}
\section{Assessments}

The main contributions of the Assessment feature include:

\begin{enumerate}
	\item Providing convenient quiz creation, usage and submission
	\item Offering useful tools for both staff and students
	\item Aiding students in revising and improving by showing explanations for questions
\end{enumerate}

\subsection{Challenges}
The main challenges when implementing the Assessments feature was creating a generic data structure that allows multiple question types to be supported, and providing the ability to change the question type (when creating a question) without causing unexpected issues. This slowed down the progress and as a result, left no time in implementing the re-usability portion of the Quiz feature, and the Poll feature altogether. Due to this, the data structure was re-designed multiple times before it was finalised and was working consistently. This resulted in significant modifications to the Quiz API endpoints being done later than expected (roughly midway through Term 3) and due to technical issues with API requests always being rejected, there was not enough time to implement it into the final implementation.

\subsection{Future Work}
Potential future works for this thesis include:

\begin{enumerate}
	\item Ability to import existing questions 
	\item Add more question types
	\item Create a statistics page on student attempts
	\item More extensive feedback when a student views their submission
	\item Integration with other features
	\item Improve user interface to be more appealing and easier to use
\end{enumerate}    
\section{Forums}

As mentioned in the project approach, the main goal of the forums component is for it to be the central location for conversation and discussion between students and educators for each course.
The aim was to implement all the required and desired features into the LMS' forum so that staff would no longer have to turn to a third-party forum services.
Aside from the standard features of viewing and adding posts, comments and replies, the additional features built that helped contribute to this goal include:

\begin{itemize}
    \item Advanced search, filtering and sorting capabilities to assist in finding specific posts;
    \item Methods for both students and staff to assist in bringing attentions to posts, like upvoting, endorsing and pinning posts; and,
    \item The ability to copy a direct link to a post to assist in sharing.
\end{itemize}

\subsection{What Would You Have Done Differently?}

If this project were to be re-done, the main thing that would've been done differently is conducting more research on different third-party libraries instead of going ahead with the first one found.
For React in particular, there are many third-party packages that essentially provide the same service but with different features.
Researching would allow for the package with the most desired features to be chosen.
Choosing a package with sufficient documentation could also prevent developers from wandering around in the dark when trying to implement it into the project.
Overall, choosing the correct package from the start would've prevented a lot of refactoring of the code, therefore saving time in the development process and allowing more features to be built on time.

\subsection{Future Work}
Future work for the forums component includes:

\begin{itemize}
    \item Ability to attach documents and files to posts;
    \item Ability to add attachments to replies and comments;
    \item More share options so that links can be shared directly to other platforms;
    \item Introduce natural language processing to parse post titles and descriptions as they are being written and generate suggested tags; and,
    \item Advanced querying that searches existing posts as users type out titles and descriptions and suggests similar posts to reduce duplicates.
\end{itemize}
\section{Gamification}

Overall, the gamification feature's design, implementation and testing phases have been successful. Taking a role based approach to requirements have prioritised users throughout the implementation phase. Throughout the thesis, there were many design and implementation considerations and tradeoffs that had to be made. The decisions made have always prioritised user experience and the gamification principles outlined in the design phase. The gamification feature has been completed on time according to timeline outlined in Thesis B. There were extra unexpected work that caused time constraints during the integration phase, as a result, a couple of lower priority requirements were removed from scope for this thesis and classified under future work. The feedback received from both students and admins have been extremely positive and has made the implementation of gamification in learning systems worthwhile.

\subsection{What would you have done differently}
\begin{itemize}
  \item Prioritised the requirements more specifically upfront and have contingency plans due to time constraints
  \item Provide more buffer time to tackle unexpected requirements that emerge during implementation
  \item Get user feedback throughout the implementation instead of just at the end to help guide some of the features more clearly
  \item Started integration work with other features early to ensure that features are more seamless
  \item Plan time according to priorities better and reuse more components to reduce the implementation effort
\end{itemize}


\subsection{Future work}
\begin{itemize}
  \item Customise Rewards like badges and achievements for students
  \item Coding game type in levels
  \item Limit number of times a level can be played
  \item Provide prescriptive analytics across multiple topic groups
  \item Allow admins to view statistics of their levels
  \item More item types in the shop
\end{itemize}
\section{Backend}
The contributions of the backend component include:
\begin{enumerate}
    \item Establishing the server side of the Meta LMS
    \item Assisting users in understanding endpoints
    \item Storing data for users and students
\end{enumerate}

\subsection{What would you have done differently}
For the backend, a more refined file hierarchy would have helped save time in refactoring and fixing code. This is because previously all of the code was in one file which eventually became large and difficult to maintain and fix. Additionally, utilising query optimisation techniques and better structuring of tables would allow for better performance in the backend.

\subsection{Future Work}
Some future work for the backend component would include:
\begin{enumerate}
    \item Scaling the backend to account for more users in the system
    \item Optimisation to improve backend performance
\end{enumerate}

\section{Overall Conclusion}
The LMS thesis project aimed to create a meta LMS that corrects the problems found in current LMS’s 
in order to improve the educative experience for instructors and students. That is, the meta LMS will 
offer reusable content, better course management features, an improved user experience to further promote 
and enhance the teaching and learning experience for users. As discussed in the report, this LMS will include 
some quality-of-life features such as a search bar, polished UI, and a feature to reuse course materials in order 
to achieve the aims outlined in the report.


\subsection{What would we have done differently - DevOps}
The primary issue that arose during development that we would have done differently if given the chance, would be to force more rigorous DevOps practices to ensure the project remains on track during the entire development, and that all developers are on the same page over the course of the development timeline. There are a number of key ways we could have improved this proccess and they are as follows:

\begin{itemize}
    \item Adopting a specific agile development methodology such as the trunk based approach \cite{trunk}
    \item Incorporating regular, development focused stand-ups into the development process
    \item Having a project management tool such as Jira or Trello to ensure a much more streamlined development of the project
\end{itemize}

\textbf{Trunk-based Development}

A key area where we struggled during development was that due to the turbulent nature of the teams schedules, often development work would take long periods of time. An unfortunate side-effect of this is that this caused some feature branches to exist for a lot longer than expected, and by the time the feature work was completed and the branches merged back into the master branch, they were extremely out of date with other portions of the app, leading to a disproportionate amount of development time being spent resolving merge conflicts and performing testing within other modules, rather than on developing new features that enhance the LMS. The trunk-based development approach \cite{trunk} enforces guidelines that force feature branches to be smaller, and not exist for too long before being merged back into master. This means that feature work is done in small chunks over one or two days rather than in large components taking multiple weeks making them not only less likely to break the app on merge, but also making them much easier to test. Another core component of the trunk-based approach is the inclusion of pull requests (PRs) into the development cycle. By incorporating PRs in our development, it would allow for multiple team members to independently test and verify each feature, leading to less bugs and errors, as well as compatibility issues down the track. Using a development approach like this would have greatly increased the productivity of the team over the course of development, potentially leading to a more feature rich product.

\textbf{Regular Stand-ups}

Over the course of development, we had weekly meetings to discuss the overall progress of the project. While this was an extremely beneficial tool in ensuring consistent progress was being made, this meeting were not particularly development focused, and rather focused more holistically on the project. The inclusion of specifically development focused stand-ups in the development process may have allowed us to more efficiently allocate development time to certain features if they were falling behind, or identify potential application wide issues and make application wide decisions far early in the process. This may have allowed us to more efficiently manage our time and resources, and perhaps deliver a more feature rich and well integrated product.

\textbf{Project Management Tools}

The final area in which we would have done things differently is the use of a project management tool such as Jira to more clearly visualise both the progress of the project while it was in development, and the work still needing to be completed. Over the course of the project, we split the required work by feature. Instead of this, we could have treated these features as epic stories, and broken these epic stories into smaller user stories representing pieces of development work. These user stories could then very easily be catalogued within a tool such as Jira as tickets. This would have allowed us to then break up the work based on each team members skill sets, assigning tickets to people based on the type of development it involved rather than the feature they existed within. This may have potentially increased the speed of development due to less time needed to be spent up-skilling over the course of the project.


\subsection{What would we have done differently - Technology}
While the technology stack used successfully delivered a highly functional LMS, during development it became apparent that we made a few decisions on technologies to be used that while they were functional, were not the optimal solution for our requirements. These two main technology decisions we would change if we were to do the project again are as follows:

\begin{itemize}
    \item Using TypeScript instead of vanilla JavaScript for our primary programming language
    \item Using MongoDB instead of PostgreSQL for our database
\end{itemize}

\textbf{Typescript}

While the type-less nature of JavaScript makes it extremely flexible and easy to use, it also leaves it prone to many type errors that using other languages avoid. TypeScript, however, gives JavaScript the benefits of strict typing while still keeping the benefits of its ease of use and power on the web. By using TypeScript in our project, we could have greatly reduced the time spent fixing type related errors, while also being able to more strongly define the transfer of different types of data between the front-end and back-end using types. Although this approach would have required a lot of the team to spend time learning how to use TypeScript, this time would most likely be less than the time spent instead diagnosing and fixing type related errors over the course of the project, making TypeScript a great choice if we were to complete this project again.

\textbf{MongoDB}

Although PostgreSQL is a very powerful database and it worked effectively throughout the course of the project, a lot of development time on the back-end was spent formatting both the data we recieved from the database to be compatible with out JavaScript application, and formatting the data generated by our application to be compatible with the database. MongoDB, on the other hand, uses a JSON-like data format that is highly compatible with JavaScript applications that would have greatly reduced development time for database interactions. Much like TypeScript, the trade-off required for time spent up-skilling to use MongoDB instead of PostgreSQL would most likely have been less than the time it would save not having to spend time converting the data sent to and recieved from our PostgreSQL database.

\subsection{Additional features}
The features listed in this section are are additional features that will be implemented in the future. These features have been compiled into a list as below: 

\begin{itemize}
  \item Assignments
  \item Attendance and Grading
  \item Blogs/Wikis/Discussions
  \item Notifications
  \item Third Party Integration/Data Migration
  \item Inbox/Messaging
\end{itemize}
