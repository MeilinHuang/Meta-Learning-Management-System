\chapter{Introduction}\label{ch:intro}

\section{Motivation}
A learning management system is a software system that administers, documents, facilitates and tracks the progress of educational courses, training programs, and more. They are mostly deployed by educational institutions such as universities and schools, but are also used at corporate institutions to facilitate training programs\cite{lmsdefinition}. A learning management system helps academics create and organise course materials, track students' progress, improves communication between academics and students and helps facilitate remote learning, which is especially important with the pandemic \cite{lmsreasons}. Students will have easier access to course resources and can also track what assessments are due, as well as communicate with other students of the same course.\\

This thesis will focus on learning management systems (LMS) that are deployed at institutions. There are a wide range of learning management systems available, such as Moodle, WebCMS, Edmodo, OpenLearning, and more that will be looked at in this thesis, however most do not allow for easy reuse and management of content and do not provide good import or export functionality of educational material. Some provide simple functionality of reusing materials from the same course in a new offering of the course, but if a new course is created from scratch it can be difficult to import material from other courses.\\

Most LMS also do not provide key features such as searchable forums or a polished UI, and so some instructors will look at using other software platforms in conjunction with the learning management system, such as Ed or Piazza. This can degrade the learning experience for students, and make it more difficult to facilitate student learning.\\

\section{Learning Management System}
This learning management system aims to provide easier reuse of content and an improved structure of content for instructors. Instructors will upload content to a ``topic``, a subject that students will learn about. For example, a COMP1511 (Introduction to Programming) instructor would upload content about pointers to a topic called ``Pointers``. Topics can have prerequisites, for example to learn the topic called pointers, students must learn about ``Memory in C``. This allows instructors to easily import content into new courses, improving reuse of content and providing structure to content, making it faster to create and manage courses.\\
The Meta LMS will be built collaboratively, with students developing their own features of the LMS and collaborating to integrate it into the system. This is explained further in the contributions section of this thesis. \\
Features such as the topic tree, exam management and third party integration will help improve course management for instructors. Similarly, features such as gamification, assessment notifications and forums also help improve students' learning by facilitating student learning.\\

\section{Aims}
The aims for developing a meta LMS are to provide better flexibility, utility and usability for both course instructors and students. 
Therefore the aims are to develop a LMS that allows instructors to easily curate courses and reuse content and provide students with a high quality learning experience. 
This report has considered a comprehensive set of features that we plan on implementing to achieve these aims. 
These features were developed through analysing other LMS’s which are compiled in our literature review.
The literature review provides a basis in which to analyse other competitors and understand their design principles and implementation.
Through the literature review a set of features for the designed meta LMS are planned in which further the aims for developing a meta LMS.
