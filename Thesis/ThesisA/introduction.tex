\chapter{Introduction}\label{ch:intro}

\section{Motivation}
A learning management system is a software system that administers, documents, facilitates and tracks the progress of educational courses, training programs, and more. They are mostly deployed by educational institutions such as universities and schools, but are also used at corporate institutions to facilitate training programs.\cite{lmsdefinition}\\

This thesis will focus on learning management systems (LMS) that are deployed at institutions. There are a wide range of learning management systems available, such as Moodle, WebCMS, Edmodo, OpenLearning, and more that will be looked at in this thesis, however most do not allow for easy reuse and management of content and do not provide good import or export functionality of educational material. Some provide simple functionality of reusing materials from the same course in a new offering of the course, but if a new course is created from scratch it can be difficult to import material from other courses.\\

Most LMS also do not provide key features such as searchable forums, , and so some instructors will look at using other software platforms in conjunction with the learning management system, such as 