\newpage
\section{Backend}
\subsection{Overview}
The backend of the meta learning system is arguably one of the most important components, as it is responsible for storing and 
organising data. Additionally, this component will be appropriately implemented to optimise the LMS runtimes.

The backend component will have no user interface and will only include the database, data, endpoints and documentation. In addition, 
the backend component will list all the endpoints via documentation. This component will only be accessible to developers and admins through 
a combination of both backend files and a command terminal.

Overall, the successful implementation of the backend component of the LMS will constitute the foundation of the LMS and 
ensure effective runtimes for the entire LMS system.

\subsection{Design}
There will be many database schemas which will be implemented in the backend. These schemas will be dependant on the requirements for each 
feature. For example, the schema for the lectures and tutorials feature will include relations between lectures and lecturers.  

Moreover, an example of a table structure for the lectures and tutorials will include: the lecture id, the lecture title, the 
lecture associated files, the lecturers in charge and the enrolled students in the lecture. 

\begin{figure}[h!]
    \centering
    \includegraphics[scale=0.6]{backend_sample_tables}
    \caption{Sample Database Tables}
\end{figure}

The calls that can be made to the backend will include data retrieval, updates, creation and deleting. An example of such a 
creation call is creating a lecture entry. Moreover, the calls samples will be further updated in the future to include a variety 
of calls for different schemas. For instance, a GET request for students enrolled in a course.

\begin{figure}[h!]
  \centering
  \includegraphics[scale=0.6]{backend_post_sample}
  \caption{Lecture POST Request}
\end{figure}

\newpage
\subsection{Requirements}
The requirements listed below will be in the general form as specific API calls for each 
component have not been completely finalised. Moreover, the requirements for the backend 
component will be adjusted and changed in accordance with the needs of the other components. 

\subsection{Functional Requirements}

\textbf{API}
    \begin{enumerate}
    \item Users can retrieve data
    \item Users can change data
    \item Users can delete data
    \item Users can create data
    \end{enumerate}

\textbf{Database}
    \begin{enumerate}
    \item Admins can view database entries
    \item Admins can edit database entries
    \item Admins can create database entries
    \item Admins can delete database entries
    \end{enumerate}

\subsection{Non-Functional Requirements}
  \begin{enumerate}
    \item Usability - The feature must be efficient and effective
    \item Performance - The feature must process queries within a reasonable timeframe
    \item Learnabilty - The feature must be easy to learn and use
    \end{enumerate}

\subsection{Timeline}
The timeline for the backend will outline the tasks and the completion dates for each during Thesis B and C.

\begin{figure}[h!]
    \centering
    \includegraphics[scale=0.4]{backend_thesisB_gantt}
    \caption{Thesis B Timeline}
\end{figure}

For Thesis B, the first task is to develop database schemas which are required for the other features and integrate 
it into the backend. This task is important as it is the foundation for the creation of endpoints in the next task. 

The endpoints will be on the next agenda as this is the most important feature of the backend. The features discussed 
in this LMS require endpoints in order to access, manipulate and store data for the LMS. Also, these endpoints will be 
developed as the need for specific requests arise. For instance, there may be a need to retrieve a list of lecturers teaching a course.

Furthermore, the user documentation will be an ongoing task that occurs simultaneously with the endpoint integration feature. 
The documentation is crucial to assist developers in understanding what each endpoint does, the type of request it is and an
example request body to further clarify the use of each endpoint. To reiterate, the user documentation is ongoing to ensure 
developers understand any old and new additions to the backend components. 

\begin{figure}[h!]
    \centering
    \includegraphics[scale=0.4]{backend_thesisC_gantt}
    \caption{Thesis C Timeline}
\end{figure}

In the Thesis C timeline, priority will be to develop and integrate any additional schemas if any additional features 
are incorporated into the LMS. Similarly, endpoints will also be developed and then integrated into the backend for 
these additional features. These schemas and endpoints can also be applied to new requirements for any existing features. 

The remainder of Thesis C will be utilised to finalise the final demo presentation and the report. 

\newpage
\subsection{Milestones}
The major milestones for the backend feature are outlined below: 

\begin{enumerate}
\item Week 1 Term 2 2021 – Develop database schemas and integrate new and existing schemas into the backend system
\item Week 2 Term 2 2021 – Developing and integrating necessary endpoints for each feature into the backend
\item Week 6 Term 2 2021 – Compose user documentation for the backend components.
\item Week 1 Term 3 2021 – Develop and integrate any additional endpoints
\item Week 5 Term 3 2021 – Final testing and fixes of the backend components
\end{enumerate}

\subsection{Evaluating Results}
The evaluation criterion which will be used to determine the success of the backend components in satisfying its purpose and requirements.

\begin{enumerate}
    \item Performance: Does the feature process queries within a reasonable timeframe
    \item Usability/Learnabilty: Whether the Feature is easy to learn and use
    \item Functional Requirements: Does the feature correctly satisify each requirement
\end{enumerate}