\section{Project Management}

This section details the methods and tools used to help manage this project and separate workload between the seven group members of this thesis.

\subsection{Communication}

Throughout the duration of this project, the two main platforms used for communication between group members and staff were Slack and Microsoft Teams.
In order to ensure that all discussions about the implementation, presentations and reports were kept in one place, Slack was used as the main platform for text communication.
Weekly meetings were held on Microsoft Teams and allowed each group member to explain what they had been working on the week prior and ask any questions.
Occasionally, additional student-only meetings were held to prepare for presentations or plan reports.

\subsubsection{Comunication Between Teams}

While everyone lead the development of a frontend component, there were only a couple group members who were in charge of the backend.
This meant that there needed to be a lot of discussion between group members during the backend development phase of the project.
To relieve some of the load on the backend developers when it came to designing the API and database, each of the frontend leads devised the schema required for their feature.
These schemas were comprised of backend requirements like database tables and API endpoints.
This helped speed up the backend implementation process as the developers could focus mainly on building the API instead of designing it.

Once the initial API was completed, any changes or requests for new features were sent to the backend developers via Slack so that they could be implemented.
When necessary, frontend developers could also make their own changes or add features to the backend.

\subsection{Integration}

The dashboard and course pages was the central hub when it came to integrating all of the LMS components.
Links to the Topic Tree and Gamification features were added to the sidebar on the dashboard while the course page contained links to the Accounts, Lectures and Tutorials, Assessments and Forums components.

\subsection{Code Management}
A GitLab repository was used to house all the code for this project.
Version control was essential in a group this size as it ensured that no one's work was being overwritten, accidentally deleted or changed without them knowing.
Each feature had its own branch that would be merged into master once it reached a stage in development where it was production ready.
This meant that the implemented features were complete and there were minimal bugs.
This ensured that the code in master could be deployed and used without any issues.

A similar process was used for writing the thesis reports, where each group member created their own branch for writing their section of the report.
These branches were all merged into master to produce a final copy for submission.


