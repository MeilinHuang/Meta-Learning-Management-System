\section{Lectures and Tutorials}
The lectures and tutorials component will be assessed on based on the following:

\begin{itemize}
    \item Functional Requirements
    \item Non-functional Requirements
\end{itemize}

\subsection{Functional Requirements}
The functional components for the lectures and tutorials component have been satisfactorily completed as 14 out of 14 requirements have been completed. Listed below are the functional requirements for this feature.

\textbf{Lectures and Tutorials}
    \begin{enumerate}
    \item Users can toggle lectures and tutorials subsection visibility
    \end{enumerate}

\textbf{Video Streaming}
    \begin{enumerate}
    \item Users can watch a live lecture
    \item Users can watch a live tutorial
    \item Instructors can stream a live lecture
    \item Instructors can stream a live tutorial
    \end{enumerate}

\textbf{File System}
    \begin{enumerate}
    \item Users can search for a file
    \item Instructors can upload files to tutorials subsection
    \item Instructors can upload files to lectures subsection
    \item Users can download a recorded lecture
    \item Users can download a recorded tutorial
    \item Users can download tutorial files
    \item Users can download file files
    \item Users can view tutorial files
    \item Users can view lecture files
    \end{enumerate}

\subsection{Non-Functional Requirements}
The non-functional requirements for this component have been completed to an adequate level, as 3/3 requirements have been satisfied. 

\textbf{Non-Functional Requirements}
  \begin{enumerate}
    \item Usability - The feature must be efficient and effective
    \item Performance - The feature must operate within a reasonable time frame
    \item Aesthetics - The feature must be visually pleasing
    \item Learnabilty - The feature must be easy to learn and use
    \item Accessibility - The feature must accommodate varying users
\end{enumerate}

\subsubsection{Usability}
The lectures and tutorials component is efficient and effective to use as there is a minimal amount of clicks to complete a desired task. Additionally, the component is efficient as each action is performed relatively well and quickly.

\subsubsection{Performance}
The component suffers from an occasional bug where an uploaded file does not render immediately. As a result this requirement is somewhat completed. Otherwise, the other functions such as deleting, occasional uploading, updating weeks are relatively quick in performance.

\subsubsection{Aesthetics}
The aesthetics of the component are adequately pleasing as the colours and spacing are well put. Additionally, the inclusion of icons and hover effects also add to the aesthetics of the component. However, there is room for improvement in this requirement such as allowing for different colours and fonts for the component to add for visual effects.

\subsubsection{Learnability}
The learnability requirement is sufficiently satisfied as the functionalities of the buttons are clearly labelled and inferred by the icons. For instance, the red trash icon indicates a delete functionality for the feature. Moreover, as stated in the usability tests most participants were able to complete the tasks within 5-10 seconds which indicates the learnability requirement being satisfied.

\subsubsection{Accessibility}
As for the lectures and tutorials feature, there were no features implemented that took into account accessibility. In the future, there could be additional work such as a screen reader and a font changer to accommodate all users.

\subsection{Usability Tests}
This subsection will depict the results from usability tests conducted focusing on the lectures and tutorials feature of the LMS. This was tested on 5 anonymous participants whom are familiar with learning management systems such as Moodle and Canvas.

\subsubsection{Results}
The results of the usability tests are conveyed below:

\begin{itemize}
    \item 5 out of 5 participants were able to complete all of the tasks given
    \item The time taken to complete each task was on average 5-10 seconds which indicates that the feature is easy to understand and use
\end{itemize}

\subsubsection{Feedback}
Some feedback provided by the participants were:

\begin{itemize}
    \item The lectures and tutorials components could be combined into a single page since both components were similar
    \item The placement of the add new weeks was ambiguous and could be better placed on the page
    \item There should be a visual representation of the videos in question for aesthetic reasons
    \item An in-browser document viewer would save users time if they wanted to look at only certain parts of a file
    \item Some customisation features for the pages is desired
    \item There should be a feature to drag multiple files to upload to the relevant week
    \item A file repository system for the course would be great as the goal of the Meta LMS was for re usability
\end{itemize}

\subsection{Challenges}
The challenges faced when implementing the lectures and tutorials component was understanding how to use the backend endpoints to create the frontend for this feature. As there was unfamiliarity with integrating both, this led to a delay in completing this feature. Additionally, there was also an issue with how to link to an external video provider to host lectures and tutorials. Moreover, there were some issues with using the ChakraUI on the frontend which also resulted in a delay in the feature completion date.