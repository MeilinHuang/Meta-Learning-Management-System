\section {Assessments}
Since the logic and data layer was implemented by other thesis team mates, only the presentation layer and user interface design will be mentioned for the Assessments feature.

\subsection{Presentation Layer}
The presentation layer is the layer which the user uses to interact with the system via the user interface and is where the operations that initiate data creation and modifications are located in.

The following technologies were used:
\begin{enumerate}
	\item ReactJS - a front-end JavaScript library used to create user interfaces using a component system where these components are modular and can be re-used in different pages. This library was chosen due to its usages being less complex and simpler than other libraries such as AngularJS.
	\item JavaScript - a programming language popularly used for web development and has many front-end libraries that use it. JavaScript was used since there was less experience with other languages therefore making it more advantageous in the interest of the time.
	\item React Router - a library used for routing and navigation between components in ReactJS. It allows passing data and navigation between components a lot cleaner and simpler.
	\item Chakra UI - A UI library that is simple, easy to use and provides a visually appealing interface. 
\end{enumerate}

\subsection{User Interface Design}
The user interface is what the user relies on to complete tasks and thus it is important that the design is usable and easy to use. A few changes were made from the initial designs to support the added content that was not planned in earlier stages of the Assessments feature. A more detailed insight of the user interface is shown in the Walkthrough chapter under the Assessments section. 

\subsubsection{Final Designs}
TODO

\subsubsection{Changes Made From Initial Designs}
TODO