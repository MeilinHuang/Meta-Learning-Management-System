\section{Backend}
The backend component will be assessed on based on the following:

\begin{itemize}
    \item Functional Requirements
    \item Non-functional Requirements
\end{itemize}

\subsection{Functional Requirements}
The functional requirements for the backend component have been completed. The priorities of each requirement are classified as equally important, as each category is general. 

In total 8/8 functional requirements have been completed for both generalised API and database categories. This completion of the requirements is displayed below.

\textbf{API}
\begin{enumerate}
    \item Users can retrieve data
    \item Users can change data
    \item Users can delete data
    \item Users can create data
\end{enumerate}
    
\textbf{Database}
\begin{enumerate}
    \item Admins can view database entries
    \item Admins can edit database entries
    \item Admins can create database entries
    \item Admins can delete database entries
\end{enumerate}

\subsection{Non-Functional Requirements}
The non-functional requirements for this component have been completed to an adequate level, as 3/3 requirements have been satisfied. 

\textbf{Non-Functional Requirements}
  \begin{enumerate}
    \item Usability - The feature must be efficient and effective
    \item Performance - The feature must process queries within a reasonable timeframe
    \item Learnabilty - The feature must be easy to learn and use
\end{enumerate}

\subsubsection{Usability}
From the backend each endpoint is effective at completing its task and efficient to use. The backend is efficient as users simply need to enter inputs and execute to receive the response output.

\subsubsection{Performance}
The backend does somewhat process queries within a reasonable timeframe if the size of the database is small although the backend can be improved to better match this requirement. 

\subsubsection{Learnability}
There is an OpenAPI specification that allows users to easily understand the required inputs and outputs of each endpoint. Additionally, learnability is further supported with sample responses and inputs provided on the OpenAPI specification.

\subsection{Challenges}
The challenges when implementing the backend was setting up the backend with APIs and databases, as a lack of experience made this challenging. For instance, understanding how to create Restful APIs, integration of NodeJS with PostgreSQL and also incorporating the visualisation of the APIs via OpenAPI specification. Additionally, devising a method to locally store the files on the system was also challenging as there was no usage of cloud storage methods. Although there were some challenges to the backend implementation, the backend is arguably completed and adheres to the scope of the Meta LMS.